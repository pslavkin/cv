\section{Presentación}
Soy \emph{Ingeniero Electrónico} del ITBA, recibido recientemente de \emph{Especialista
en Sistemas Embebidos} y cursando una \emph{Maestría en Sistemas Embebidos} de
la UBA. \\
Desarrollé mi carrera trabajando en el área de desarrollo de producto de varias
empresas nacionales y en el área de investigación en instituciones estatales.\\
Estuve a cargo de un estudio de ingeniería electrónica ofreciendo servicios de
diseño y producción electrónica y actualmente trabajo como desarrollador
electrónico freelance con posibilidad de emitir facturas 'A' y 'B'.\\
Trabajo diariamente diseñando equipos electrónicos embebidos ejecutando tareas como: \\
\cvlistitem{Toma de requerimientos y planificación de los test de aceptación de hard y soft.}
\cvlistitem{Diseño de esquemáticos, PCB, modelado 3D y simulaciones.}
\cvlistitem{Codificación para tiempo real en C/C++ en bare metal o sobre RTOS.}
\cvlistitem{Codificación y ejecución de los test unitarios y manejo de herramientas de integración continua.}
\cvlistitem{Armado y puesta en marcha de prototipos y documentación para la Línea de montaje.}
Soy muy pragmático, comprometido y disfruto resolver los problemas complejos de
modo creativo intercambiando ideas con mis pares. Prefiero los desarrollos
down-top utilizando conceptos ágiles para mantener el producto funcional desde
el inicio.\\
Cuento con un taller de electronica con herramientas tales como: \\
\cvlistitem{Línea de montaje de placas SMD y TH, stencil de pasta, pick and place, horno de refusión y batea.}
\cvlistitem{Herramientas de reworking y soldadura manual}
\cvlistitem{Stock de materiales SMD y TH de uso corriente y específicos.}
\cvlistitem{Centro de mecanizado CNC.}
\cvlistitem{Máquina para corte y grabado laser.}
\cvlistitem{Varias maquinas para impresión 3D.}
\cvlistitem{Generadores, Osciloscopios e Instrumental avanzado para medición y diagnóstico.}
\cvlistitem{Herramientas electrónicas para desarrollo de firmware.}
Estas herramientas, mi experiencia, capacidad técnica y frecuente actualización académica me
permiten desenvolverme en la mayoría de las instancias del desarrollo de un
equipo electrónico embebido profesional.\\
