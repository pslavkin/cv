\section{Experience}
   \subsection{\bfseries{Professional}}
   \cventry { 2021--Present}{Embedded firmware and software developer           }{\href{\linkpalrobotics              }{ PAL Robotics      }  }{Barcelona, España           }{}{ I work as a senior embedded firmware and software developer for the boards utilized on the robots. I work on the field of motor control loops, multicore SoC's bootloaders, real time operatig systems and bring-up of new boards. I work with Agile driven, multidisciplinar group and exciting technologies. \hyperlink{subsec:pal_robotics}{See portfolio.}}
   \cventry { 2020--2021}   {Lead Embedded Software Engineer                    }{\href{\linknovo                     }{ Novo Space        }  }{EE.UU, Argentina            }{}{ As the first employee of the start-up, I worked as low level firmware, real time OS, bootloaders mainly u-boot, embedded Linux and bring up of new complex hardware boards. I work remotely for 12 month, the company grew up to more than 15 employees. \hyperlink{subsec:novo_space}{See portfolio.}} %, but I quit 'cause relocation. 
   \cventry { 2020--2020}   {Design and development of BLDC power stage         }{\href{\linkengarts                  }{ Engineered Arts   }  }{England                     }{}{ I work as a hardware engineer, making the electronic design, choosing componentes and topology for the power stage of a new BLDC servo drive. I work remotely with a team of specialists. \hyperlink {subsec:engineered_arts}{ See portfolio. }}
   \cventry { 2019--Present}{Freelance Electronic Engineer                      }{                                                            }{                            }{}{ Personal entrepreneurship Electronic design services, hardware, firmware and electronic equipment.                                                                                                                }
   \cventry { 2019--2020}   {Development of a PMSM servomotor controller        }{Nanocut                                                     }{Chisináu, Moldavia          }{}{ For a company in the industrial machinery field, I work in the development of an integrated servo controller for a permanent magnet synchronous motor. It'll be used for the improvement of the actual machinery. \hyperlink{subsec:nanocut}{See portfolio.}}
   \cventry { 2019--2019}   {Consultant and CNC software development            }{\href{www.wolfcut.es}{Wolfcut}                              }{Valencia, España            }{}{ I worked in the implementation of a production line management software tool. I've also developed a plugin for improve the capabilities of the CNC software, adding an automatic tool changer, an automatic tool measurement, and others features. \hyperlink{subsec:wolfcut}{See portfolio.} }
   \cventry { 2011--2019}   {Development and production of electronic equipment }{\href{ www.gruponoto.com            }{ Grupo Noto        }  }{Argentina                   }{}{ I develop and manufacture a whole line of aesthetic electro medicine electronics equipment, hardware, firmware and production. \hyperlink                                                                                                                                                                                                                                                                                                                                   {subsec:noto}{ See portfolio. } }
   \cventry { 2012--2019}   {Development and production of electronic equipment }{\href{ www.piscinanatural.com       }{ Piscina Natural   }  }{Argentina                   }{}{ In conjunction with the company was developed a system for the generation of chlorine from saline water was developed to keep the pools clean. \hyperlink                                                                                                                                                                                                                                                                                                                   {subsec:piscina}{ See portfolio. } }
   \cventry { 2011--2016}   {Consultant and developer of electronic equipment   }{\href{ www.softron.biz              }{ Softron           }  }{Argentina                   }{}{ Consulting and development of electronic equipment and solutions for energy measurement and monitoring using Zigbee wireless and GSM technologies. \hyperlink                                                                                                                                                                                                                                                                                                               {subsec:softron}{ See portfolio. } }
   \cventry { 2011--2017}   {Consultant and developer of electronic equipment   }{\href{ www.grupokoner.com           }{ Grupo Koner       }  }{Argentina                   }{}{ Consulting and development of equipment and electronic solutions for the automatic vehicle location, AVL. I worked mainly in the development and integration of an RFID card reader for drivers registration. \hyperlink                                                                                                                                                                                                                                                    {subsec:koner}{ See portfolio. } }
   \cventry { 2011--2014}   {Consultant and developer of electronic equipment   }{\href{ www.seconsat.com             }{ Seconsat          }  }{Argentina                   }{}{ Consulting and development of electronic accessories for the AVL business. I work mainly in a new multi sensor wireless dongle for AVL integration. \hyperlink                                                                                                                                                                                                                                                                                                              {subsec:seconsat}{ See portfolio. } }
  \cventry { 2006--2007     }{ Developer of wireless hearing aids }{ \href {                              }{ VBA Argentina S.R.L. }, Argentina         }{                                                                                                      }{ }{ Design and development of innovative wireless hearing implants. A set of wireless prototypes with high-fidelity digital audio capabilities, analog capture, processing, equalization, amplification and playback were designed and implemented. A software was created so that the medical staff can accurately adjust all the parameters of the equipment. \hyperlink{subsec:vba}{See portfolio.} }
   \cventry { 2005--2019}   {Director in engineering company                    }{\href{ www.disenioconingenio.com.ar }{ disenioconingenio }  }{Argentina                   }{}{ Personal entrepreneurship Engineering study that offers electronic design services to companies, with ability to develop and manufacture electronic equipment, hardware, firmware, software, mechanics, PCB routing, assembly of PCB's SMD and TH, 3D printing, CNC machining, laser cutting and engraving and commercialization of equipment for access control RFID, monitoring of Ethernet temperature, automation of machines, converters of protocols, etc. \hyperlink {subsec:dci}{ See portfolio. } }
   \cventry { 2003--2005}   {Electronic equipment developer                     }{\href{ www.digicard.com.ar          }{ Digicard          }  }{Argentina                   }{}{ Company referring to the national level in the area of access control. I've Worked on the development of an RFID reader of 125khz for the line of access controllers. I participated in all the stages since the requirements request, schematic design, PCB layout, prototype, start-up, firmware, and production documentation The reader is actively actively marketed by the company. \hyperlink                                                                      {subsec:digicard}{ See portfolio. } }
   \cventry { 2002--2003}   {Firmware developer for microcontrollers            }{\href{ www.pump-control.com.ar      }{ Pump-Control      }  }{Argentina                   }{}{ Company dedicated mainly to the design, development and production of electronic controllers for the distribution of hydrocarbons. I've worked on the area of firmware development for 8bit microcontrollers of the Atmel line, implementing 1-Wire communication protocols, access control and dispenser control fuel.                                                                                                                                                                     }
   \vfill{}

   \subsection{\bfseries{Teaching}}
   \cventry { 2020--2021 }{ Digital signal processing, introduction course, upgraded version}{\href{\linkuba}{University of Buenos Aires, UBA}} {}{}{Within the framework of the Master in Embedded Systems of the UBA, MSE, a course on digital signal processing applied to embedded systems was taught, including subjects such as: quantization, convolution, correlation, discrete Fourier transform (DFT,FFT).\href{\linkmse}{See program.}\href{\linkmsepsfvideos} {See recorded classes.}\href{\linkmsepsftwentyonematerial}{See material's course}}
   \cventry { 2020--2020 }{ Digital signal processing, introduction course}{\href{\linkuba}{University of Buenos Aires, UBA}} {}{}{Within the framework of the Master in Embedded Systems of the UBA, MSE, a course on digital signal processing applied to embedded systems was taught, including subjects such as: quantization, convolution, correlation, discrete Fourier transform (DFT,FFT).\href{\linkmse}{See program.}\href{\linkmsepsfvideos} {See recorded classes.}\href{\linkmsepsfmaterial}{See material's course}}
   \cventry { 2017--2017 }{ Introduction to robotics }{ Siglo XXI School }{}{}{ A day of introduction to robotics was given for students from the third to fifth year, showing the history, basic concepts and culminating with a practice in different commercial platforms  \href{\linkrobotsiglo}{ See certificate.}}{}
   \cventry { 2004--2004 }{ Altera FPGA programming intensive course using Quartus II }{ ITBA }{}{}{ An introductory course with practical activities was carried out using an Altera evaluation board. \href { \linkfpgasig }{ See material. } }
   \vfill{}

   \subsection{\bfseries{Research}}
   \cventry { 2015--2016 }{ Scholar in the National Atomic Energy Commission}{\href{www.cnea.gob.ar}{CNEA}}{}{}{ I worked as a fellow in the completion of a fully developed PET (Positron Emission Tomography) in the center on which the doctoral thesis plan is developed. Particularly, work is done in the area of acquisition and processing of digital signals on high performance FPGA. The scholarship is terminated doubt as a move to another city. \hyperlink{subsec:cnea}{See portfolio},  \href{\linkcneaa} { see material 2015 }, \href{\linkcneab}{ see material 2016. } }
   \cventry { 2009--2009 }{Assistant in the Research Center of Lasers and Applications}{\href{www.citedef.gob.ar}{CITEDEF}}{}{}{ I worked as an assistant of Dr. Jorge Codnia and Lic. Laura Azcárate in the assembly of a flow condenser, which with the help of a laser produces isotopes of interest, and the first advances in a new mass spectrometer of flight time. \href \linkcitefa {See material.} }
   \vfill{}

   \subsection{\bfseries{Tutorials and jury}}
   \cventry { 2021--2021}{Master thesis jury of Esp. Lic. Leopoldo A. Zimperz in his work}{\textit{Easy installation access control with remote administration.}} {\href{\linkuba}{Universidad de Buenos Aires, UBA}} {}{ Within the framework of the thesis defenses of the \emph {Master's Degree in Embedded Systems of the UBA, MSE}, I participated as a master's thesis jury. \href {\linkleopoldolimperz} {see thesis}, \href \linkleopoldolimperzpresentacion {see presentation}, \href \linkleopoldolimperzcertificado {see certificate}.}
      \vfill{}
   %\subsection{\bfseries{Otros Trabajos}}
   %\cventry{1994--2001}{Instalación y mantenimiento de instalaciones eléctricas comerciales}   {}       {}{}{Se realizaron trabajos de electricidad, instalaciones eléctricas comerciales, reparaciones y mantenimiento general a clientes particulares}


%
