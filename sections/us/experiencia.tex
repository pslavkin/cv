\section{Experience}
   \subsection{\bfseries{Professional}}
   
   \cventry { 2019--Present }{ Freelance Electronic Engineer                    }{                                                              }{ }{ }{ Personal entrepreneurship Electronic design services, hardware, firmware and electronic equipment. }
   \cventry { 2011--Presente }{ Development and production of electronic equipment}{ \href { www.gruponoto.com }{ Grupo Noto } }{ }{ }{ I develop and manufacture a whole line of aesthetic electromedicine electronics equipment, hardware, firmware and production. \hyperlink{subsec:noto}{ See portfolio. } }
   \cventry { 2012--Presente }{ Development and production of electronic equipment}{ \href { www.piscinanatural.com }{ Piscina Natural } }{ }{ }{ In conjunction with the company was developed a system for the generation of chlorine from saline water was developed to keep the pools clean. \hyperlink{subsec:piscina}{ See portfolio. } }
   \cventry { 2011--2016 }{ Consultant and developer of electronic equipment }{ \href { www.softron.biz }{ Softron } }{ }{ }{ Consulting and development of electronic equipment and solutions for energy measurement and monitoring using Zigbee wireless and GSM technologies. \hyperlink{subsec:softron}{ See portfolio. } }
   \cventry { 2011--2017 }{ Consultant and developer of electronic equipment }{ \href { www.grupokoner.com }{ Grupo Koner } }{ }{ }{ Consulting and development of equipment and electronic solutions for the automatic vehicle location, AVL. I worked mainly in the development and integration of an RFID card reader for drivers registration. \hyperlink{subsec:koner}{ See portfolio. } }
   \cventry { 2005--2019    }{ Director in engineering company                  }{ \href { www.disenioconingenio.com.ar }{ disenioconingenio }  }{ }{ }{ Personal entrepreneurship Engineering study that offers electronic design services to companies, with ability to develop and manufacture electronic equipment, hardware, firmware, software, mechanics, PCB routing, assembly of PCB's SMD and TH, 3D printing, CNC machining, laser cutting and engraving and commercialization of equipment for access control RFID, monitoring of Ethernet temperature, automation of machines, converters of protocols, etc. \hyperlink{subsec:dci}{See portfolio.} }
   \cventry { 2011--2014    }{ Consultant and developer of electronic equipment }{ \href { www.seconsat.com             }{ Seconsat          }  }{ }{ }{ Consulting and development of electronic accessories for the AVL business. I work mainly in a new multi sensor wireless dongle for AVL integration. \hyperlink{subsec:seconsat}{See portfolio.}}
   \cventry { 2003--2005    }{ Electronic equipment developer                   }{ \href { www.digicard.com.ar          }{ Digicard          }  }{ }{ }{ Company referring to the national level in the area of access control. Work was done on the development of an RFID reader of 125khz for the line of access controllers. I participated in all the stages since the requirements request, schematic design, PCB layout, prototype, start-up, firmware, and production documentation The reader is actively actively marketed by the company. \hyperlink{subsec:digicard}{See portfolio.}}
   \cventry { 2002--2003    }{ Firmware developer for microcontrollers          }{ \href { www.pump-control.com.ar      }{ Pump-Control      }  }{ }{ }{ Company dedicated mainly to the design, development and production of electronic controllers for the distribution of hydrocarbons. Work was done in the area of firmware development for 8bit microcontrollers of the Atmel line, implementing 1-Wire communication protocols, access control and dispenser control fuel.                                                                                                                                                                          }

   \subsection{\bfseries{Teaching}}
   \cventry { 2017--2017 }{ Introduction to robotics }{ Siglo XXI School }{ }{ }{ A day of introduction to robotics was given for students from the third to fifth year, showing the history, basic concepts and culminating with a practice in different commercial platforms  \href{\linkrobotsiglo}{ See certificate. } }{ }
   \cventry { 2004--2004 }{ Altera FPGA programming intensive course using Quartus II }{ ITBA }{ }{ }{ An introductory course with practical activities was carried out using an Altera evaluation board. \href { \linkfpgasig }{ See material. } }

   \subsection{\bfseries{Research}}

   \cventry { 2015--2016 }{ Scholar in the National Atomic Energy Commission }{ \href { https://www.cnea.gob.ar    }{ CNEA    } }{ }{ }{ 
      I worked as a fellow in the completion of a fully developed PET (Positron Emission Tomography) in the center on which the doctoral thesis plan is developed. Particularly, work is done in the area of acquisition and processing of digital signals on high performance FPGA. The scholarship is terminated doubt as a move to another city. \hyperlink{subsec:cnea}{See portfolio},  \href{\linkcneaa} { see material 2015 }, \href{\linkcneab}{ see material 2016. } }

\cventry { 2009--2009 }{ Assistant in the Research Center of Lasers and Applications }{ \href{http://www.citedef.gob.ar/ }{ CITEDEF } }{ }{ }{ He worked as an assistant of Dr. Jorge Codnia and Lic. Laura Azcárate in the assembly of a flow condenser, which with the help of a laser produces isotopes of interest, and the first advances in a new mass spectrometer of flight time. \href{\linkcitefa }{See material.} }

   %\subsection{\bfseries{Otros Trabajos}}
   %\cventry{1994--2001}{Instalación y mantenimiento de instalaciones eléctricas comerciales}   {}       {}{}{Se realizaron trabajos de electricidad, instalaciones eléctricas comerciales, reparaciones y mantenimiento general a clientes particulares}


%
