\section{Experience}
   \subsection{\bfseries{Professional}}
   \cventry { 2019--Present }{ Freelance Electronic Engineer                    }{                                      }{                   }{ }{ Personal entrepreneurship Electronic design services, hardware, firmware and electronic equipment. }
   \cventry { 2005--2019    }{ Director in engineering company                  }{ \href { www.disenioconingenio.com.ar }{ disenioconingenio }  }{                                                                                                    }{ }{ Personal entrepreneurship Engineering study that offers electronic design services to companies, with ability to develop and manufacture electronic equipment, hardware, firmware, software, mechanics, PCB routing, assembly of PCB's SMD and TH, 3D printing, CNC machining, laser cutting and engraving and commercialization of equipment for access control RFID, monitoring of ethernet temperature, automation of machines, converters of protocols, etc. }
   \cventry { 2011--2014    }{ Consultant and developer of electronic equipment }{ \href { www.seconsat.com             }{ Seconsat          }  }{                                                                                                    }{ }{ Consulting and development of electronic accessories for the AVL business. see portfolio \hyperlink { subsec:seconsat                                                                                                                                                                                                                                                                                                                                            }{ ver portfolio } }
   \cventry { 2003--2005    }{ Electronic equipment developer                   }{ \href { www.digicard.com.ar          }{ Digicard          }  }{                                                                                                    }{ }{ Company referring to the national level in the area of access control. Work was done on the development of an RFID reader of 125khz for the line of access controllers. He participated in all the stages since the request, design, layout, prototype, start-up, firmware, general documentation and for production. At present It is a product actively marketed by the company.                                                                               }
   \cventry { 2002--2003    }{ Firmware developer for microcontrollers          }{ \href { www.pump-control.com.ar      }{ Pump-Control      }  }{                                                                                                    }{ }{ Company dedicated mainly to the design, development and production of electronic controllers for the distribution of hydrocarbons. Work was done in the area of firmware development for 8bit microcontrollers of the Atmel line, implementing communication protocols, access control, dispenser control fuel, etc.                                                                                                                                             }

   \subsection{\bfseries{Teaching}}
   \cventry { 2017--2017 }{ Introduction to robotics }{ Siglo XXI }{ }{ }{ A day of introduction to robotics was given for students from the third to fifth year, showing the history, basic concepts and culminating with a practice in different commercial platforms }{ }
   \cventry { 2004--2004 }{ Altera FPGA programming intensive course using Quartus II }{ ITBA }{ }{ }{ An introductory course with practical activities was carried out using an Altera evaluation board. \href { \linkfpgasig }{ ver material } }

   \subsection{\bfseries{Research}}

      \cventry { 2015--2016 }{ Scholar in the National Atomic Energy Commission }{ \href { https://www.cnea.gob.ar    }{ CNEA    } }{ }{ }{ 
I worked as a fellow in the completion of a fully developed PET (Positron Emission Tomography) in the center on which the doctoral thesis plan is developed. Particularly, work is done in the area of acquisition and processing of digital signals on high performance FPGA. The scholarship is terminated doubt as a move to another city.
\href{ \linkcneaa } { ver material 2015 }, \href{ \linkcneab }{ ver material 2016 } }

      \cventry { 2009--2009 }{ Ayudante en el Centro de investigaciones de Láseres y Aplicaciones }{ \href { http://www.citedef.gob.ar/ }{ CITEDEF } }{ }{ }{ Se trabajó como ayudante del Dr. Jorge Codnia y la Lic. Laura Azcárate en el armado de un condensador de flujos, que con la ayuda de un láser produce isótopos de interés, y los primeros avances en un nuevo espectrómetro de masas de tiempo de vuelo}

   %\subsection{\bfseries{Otros Trabajos}}
   %\cventry{1994--2001}{Instalación y mantenimiento de instalaciones eléctricas comerciales}   {}       {}{}{Se realizaron trabajos de electricidad, instalaciones eléctricas comerciales, reparaciones y mantenimiento general a clientes particulares}


%
