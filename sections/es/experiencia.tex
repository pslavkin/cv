\section{Experiencia}
   \subsection{\bfseries{Profesional}}
      \cventry { 2020--Presente}{Diseño de un servo control para BLDC}{\href{\linkengineeredarts}{Engineered Arts},England}{}{}{Trabajo como ingeniero de hardware, seleccionando los componentes, la topologia y realizando el ruteo del PCB. Se trabaja remotamente junto a un equipo de especialistas.\hyperlink{subsec:engineered_arts}{Ver portfolio.}}
      \cventry { 2019--Presente }{ Ingeniero electrónico freelance                   }{                                                                                         }{                   }  { }{ Emprendimiento personal. Servicios de diseño electrónico, hardware, firmware y equipos electrónicos.                                               }
      \cventry { 2019--Presente }{ Desarrollo de un controlador para un Servomotor   }{ \href                                                    { www.nanocut.com              }{ Nanocut           }  }{ Chisináu, Moldavia                                                                                                                                 }{ }{ Para un emprendimiento de mejora de maquinaria industrial del rubro de mecanizados, se desarrolla un controlador para servomotor con motores PMSM. \hyperlink{subsec:nanocut}{Ver portfolio.}}
      \cventry { 2019--2019     }{ Consultor y desarrollador de software CNC         }{ \href                                                    { www.wolfcut.es               }{ Wolfcut           }  }{ Valencia, España                                                                                                                                   }{ }{ En la fabrica de maquinas CNC, se desarrollan trabajos de consultoria en gestion de la produccion y desarrollo de software para mejorar las capacidades tecnicas de las maquinas CNC comercializads, entre estos, un sistema de cambio de herramientas automatico y sensado de altura de herramientas.\hyperlink{subsec:wolfcut}{Ver portfolio.}}
      \cventry { 2011--2019     }{ Desarrollo y producción de equipos electrónicos   }{ \href                                                    { www.gruponoto.com            }{ Grupo Noto        }  }{                                                                                                                                                    }{ }{ Se desarrollan y fabrican multiples equipos para el rubro de electromedicina estetica.\hyperlink{subsec:noto}{Ver portfolio.}}
      \cventry { 2012--2020     }{ Desarrollo y producción de equipos electrónicos   }{ \href                                                    { www.piscinanatural.com       }{ Piscina Natural   }  }{                                                                                                                                                    }{ }{ Se desarrollo un equipo para la generación de cloro a partir de agua salina permitiendo mantener limpia las piscinas.\hyperlink{subsec:piscina}{Ver portfolio.}}
      \cventry { 2005--2019     }{ Director en empresa de ingeniería                 }{ \href                                                    { www.disenioconingenio.com.ar }{ disenioconingenio }  }{                                                                                                                                                    }{ }{ Emprendimiento personal. Estudio de ingeniería que ofrece servicios de diseño electrónico a empresas, con capacidad para desarrollar y fabricar equipos electrónicos, hardware, firmware, software, mecánica, ruteo de PCB's, montaje de PCB's SMD y TH, impresión 3D, mecanizado CNC, corte y grabado laser y comercializa equipos para control de accesos RFID, monitoreo de temperatura ethernet, automatización de maquinas, conversores de protocolos, etc. \hyperlink{subsec:dci}{Ver portfolio.}}
      \cventry { 2011--2014     }{ Consultor y desarrollador de equipos electrónicos }{ \href                                                    { www.seconsat.com             }{ Seconsat          }  }{                                                                                                                                                    }{ }{ Consultoría y desarrollo de accesorios electrónicos para el rubro de rastreo vehicular, AVL. Se trabajó en el desarrollo de soluciones inalámbricas embebidas.\hyperlink{subsec:seconsat}{Ver portfolio.}}
      \cventry { 2011--2016     }{ Consultor y desarrollador de equipos electrónicos }{ \href                                                    { www.softron.biz              }{ Softron           }  }{                                                                                                                                                    }{ }{ Consultoría y desarrollo de equipos y soluciones electrónicas para el rubro de medición y monitoreo de energía utilizando tecnologías inalámbricas y GSM. \hyperlink{subsec:softron}{Ver portfolio.}}
      \cventry { 2011--2017     }{ Consultor y desarrollador de equipos electrónicos }{ \href                                                    { www.grupokoner.com           }{ Grupo Koner       }  }{                                                                                                                                                    }{ }{ Consultoría y desarrollo de equipos y soluciones electrónicas para el rubro de rastreo vehicular, AVL. Se trabajo principalmente en el desarrollo e integración de un lector de tarjetas RFID para el registro de conductores.\hyperlink{subsec:koner}{Ver portfolio.}}
      \cventry { 2003--2005     }{ Desarrollador de equipos electrónicos             }{ \href                                                    { www.digicard.com.ar          }{ Digicard          }  }{                                                                                                                                                    }{ }{ Empresa referente a nivel nacional en el rubro de control de accesos. Se trabajo en el desarrollo de un lector RFID de 125khz para la linea de controladores de accesos. Se participó en todas las etapas desde el requerimiento, diseño, layout, prototipo, puesta en marcha, firmware, documentación general y para producción. Actualmente es un producto comercializado activamente por la empresa.\hyperlink{subsec:digicard}{Ver portfolio.}}
      \cventry { 2002--2003     }{ Desarrollador de firmware para microcontroladores }{ \href                                                    { www.pump-control.com.ar      }{ Pump-Control      }  }{                                                                                                                                                    }{ }{ Empresa dedicada principalmente al diseño, desarrollo y producción de controladores electrónicos para la distribución de hidrocarburos. Se trabajó en el área de desarrollo de firmware para microcontroladores de 8bits de la linea Atmel, implementando protocolos de comunicaciones, control de accesos, control de dispenser de combustible, etc.}

   \subsection{\bfseries{Docencia}}
   \cventry { 2020--2020}{Procesamiento de señales, introduccion}                               {\href{\linkuba}{Universidad de Buenos Aires, UBA}} {}{}{En el marco de la \emph{Maestria en Sistemas Embebidos de la UBA, MSE}, se dictó un curso de procesamiento de señales digitales  aplicado a sistemas embebidos incluyendo temas como: cuantizacion, convolucion, correlacion, transformada discreta de Fourier (DFT, FFT). \href{\linkmse}{Ver programa.} \href{\linkmsepsfvideos}{Ver clases grabadas.} \href{\linkmsepsfmaterial}{Ver material del curso}}
      \cventry { 2017--2017} { Jornada de introducción a la robótica}                               { Escuela Siglo XXI} { } { } { Se dictó una jornada de introducción a la robótica para alumnos de tercer a quinto año, mostrando las historia, conceptos básicos y culminando con una practica en diferentes plataformas comerciales.}                     { }
      \cventry { 2004--2004} { Curso intensivo de programación de FPGA de Altera usando Quartus II} { ITBA}      { } { } { Se realizó un curso introductorio con actividades practicas usando una placa de evaluación de Altera. \href { \linkfpgasig } { ver material}}

   \subsection{\bfseries{Investigación}}
      \cventry { 2015--2016 }{ Becario en la Comisión Nacional de Energía Atómica                 }{ \href {www.cnea.gob.ar    }{ CNEA    } }{ }{ }{ Se trabajó como becario en la culminación de un PET (Positron Emission Tomography) íntegramente desarrollado en el centro sobre el cual se desarrolla el plan de tesis doctoral. Particularmente se trabaja en el área de adquisición y procesamiento de señales digitales sobre FPGA de alta performance. Se termina la beca por mudanza a otra ciudad \href { \linkcneaa }{ ver material 2015 }, \href { \linkcneab }{ ver material 2016 } }
      \cventry { 2009--2009 }{ Ayudante en el Centro de investigaciones de Láseres y Aplicaciones }{\href{www.citedef.gob.ar/}{CITEDEF}}{}{}{ Se trabajó como ayudante del Dr. Jorge Codnia y la Lic. Laura Azcárate en el armado de un condensador de flujos, que con la ayuda de un láser produce isótopos de interés, y los primeros avances en un nuevo espectrómetro de masas de tiempo de vuelo}

   %\subsection{\bfseries{Otros Trabajos}}
   %\cventry{1994--2001}{Instalación y mantenimiento de instalaciones eléctricas comerciales}   {}       {}{}{Se realizaron trabajos de electricidad, instalaciones eléctricas comerciales, reparaciones y mantenimiento general a clientes particulares}


