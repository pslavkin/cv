\section{Experiencia}
   \subsection{Profesional}
      \cventry { 2019--Presente} { Ingeniero electronico freelancer}                  { } { } { } { Emprendimiento personal. Servicios de diseño electrónico, hardware y firmware para proyectos de manera independiente. }
      \cventry { 2005--2019}     { Director en empresa de ingeniería}                 { \href { www.disenioconingenio.com.ar} { disenioconingenio}} { } { } { Emprendimiento personal. Estudio de ingeniería que ofrece servicios de diseño electrónico a empresas, con capacidad para desarrollar y fabricar equipos electrónicos, hardware, firmware, software, mecánica, ruteo de PBC's, montaje de PCB's SMD y TH, impresión 3D, mecanizado CNC, corte y grabado laser y comercializa equipos para control de accesos RFID, monitoreo de temperatura ethernet, automatización de maquinas, conversores de protocolos, etc. \href { http://disenioconingenio.com.ar/producto.php?products_id=398} { ver detalles}}
      \cventry { 2011--2014}     { Consultor y desarrollador de equipos electrónicos} { \href { www.seconsat.com}             { Seconsat}}          { } { } { Consultoría y desarrollo de accesorios electrónicos para el rubro AVL. \hyperlink{subsec:seconsat}{ver portfolio}}
      \cventry { 2003--2005}     { Desarrollador de equipos electrónicos}             { \href { www.digicard.com.ar}          { Digicard}}          { } { } { Empresa referente a nivel nacional en el rubro de control de accesos. Se trabajo en el desarrollo de un lector RFID de 125khz para la linea de controladores de accesos. Se participo en todas las etapas desde el requerimiento, diseño, layout, prototipo, puesta en marcha, firmware, documentación general y para producción. Actualmente es un producto comercializado activamente por la empresa. \href                                                          { http://disenioconingenio.com.ar/producto.php?products_id=393} { ver detalles}}
      \cventry { 2002--2003}     { Desarrollador de firmware para microcontroladores} { \href { www.pump-control.com.ar}      { Pump-Control}}      { } { } { Empresa dedicada principalmente al diseño, desarrollo y producción de controladores electrónicos para la distribución de hidrocarburos. Se trabajó en el área de desarrollo de firmware para microcontroladores de 8bits de la linea Atmel, implementando protocolos de comunicaciones, control de accesos, control de dispenser de combustible, etc. \href                                                                                                            { http://disenioconingenio.com.ar/producto.php?products_id=391} { ver detalles}}

      \subsection{Docencia}
      \cventry { 2017--2017} { Jornada de introducción a la robótica}                               { Siglo XXI} { } { } { Se dicto una jornada de introducción a la robótica para alumnos de tercer a quinto año}                     { }
      \cventry { 2004--2004} { Curso intensivo de programación de FPGA de Altera usando Quartus II} { ITBA}      { } { } { Se realizó un curso introductorio con actividades practicas usando una placa de evaluación de Altera. \href { http://disenioconingenio.com.ar/shop/docs/fpga.pdf} { ver material}}

   \subsection{Investigación}
      \cventry { 2015--2016 }{ Becario en la Comisión Nacional de Energía Atómica                 }{ \href { https://www.cnea.gob.ar    }{ CNEA    } }{ }{ }{ Se trabaja como becario en la culminación de un PET (Positron Emission Tomography) íntegramente desarrollado en el centro sobre el cual se desarrolla el plan de tesis doctoral. Particularmente se trabaja en el área de adquisición y procesamiento de señales digitales sobre FPGA de alta performance. Se termina la beca por mudanza a otra ciudad \href { http://disenioconingenio.com.ar/producto.php?products_id=455 }{ ver material 2015 }, \href { http://disenioconingenio.com.ar/producto.php?products_id=456 }{ ver material 2016 } }
      \cventry { 2009--2009 }{ Ayudante en el Centro de investigaciones de Láseres y Aplicaciones }{ \href { http://www.citedef.gob.ar/ }{ CITEDEF } }{ }{ }{ Se trabajó como ayudante del Dr. Jorge Codnia y la Lic. Laura Azcárate en el armado de un condensador de flujo láser para la generación de isótopos, y los primeros avances en un nuevo espectrómetro de masas de tiempo de vuelo                                                                                                                                                                                            }

   %\subsection{Otros Trabajos}
   %\cventry{1994--2001}{Instalación y mantenimiento de instalaciones eléctricas comerciales}   {}       {}{}{Se realizaron trabajos de electricidad, instalaciones eléctricas comerciales, reparaciones y mantenimiento general a clientes particulares}


