%----------------------------------------------------------------------------------------
%	PACKAGES AND OTHER DOCUMENT CONFIGURATIONS
%----------------------------------------------------------------------------------------
\documentclass[11pt,a4paper,sans]{moderncv} 	% Font sizes: 10, 11, or 12; paper sizes: a4paper, letterpaper, a5paper, legalpaper, executivepaper or landscape; font families: sans or roman
\moderncvstyle{classic} 			% CV theme - options include: 'casual' (default), 'classic', 'oldstyle' and 'banking'
\moderncvcolor{orange} 				% CV color - options include: 'blue' (default), 'orange', 'green', 'red', 'purple', 'grey' and 'black'
\usepackage[utf8]{inputenc} 			% UTF8 para meter acentos y enies
\usepackage[scale=0.75]{geometry} 		% Reduce document margins
%----------------------------------------------------------------------------------------
%	NAME AND CONTACT INFORMATION SECTION
%----------------------------------------------------------------------------------------
\firstname 		{Pablo}
\familyname 		{Slavkin} 
\title 			{Curriculum Vitae}
\address 		{Padilla 876 3\textsuperscript{o} 5\textsuperscript{o}, Capital Federal}{Buenos Aires, Argentina}
\birth 			{13/12/1976}
\mobile 		{(+54)(911) 6 243 3463}
\phone 			{(+54)(11) 4962 6373}
\email 			{pslakin@disenioconingenio.com.ar}
\homepage 		{www.disenioconingenio.com.ar/categoria.php?css=1\&categories_id=191}{Online CV}
\photo[130pt][0.4pt] 	{pictures/cara76} 
%----------------------------------------------------------------------------------------
\begin{document}
\makecvtitle % Print the CV title
%----------------------------------------------------------------------------------------
%	EDUCATION SECTION
%----------------------------------------------------------------------------------------
\section{Educación}
\cventry{2007--Presente}{Doctorado en Ingeniería} 	{UTN - Universidad Tecnología Nacional FRBA}	{Buenos Aires} 		{\textit{Promedio -- 10 sobre 3 materias aprobadas + 3 finales adeudados}} 	{Mención Procesamiento digital de imágenes y señales} 
\cventry{1996--2005} 	{Ingeniería Electrónica} 	{ITBA - Instituto Tecnológico de Buenos Aires}	{Buenos Aires} 		{\textit{Promedio -- 6.5}} 							{}
\cventry{1990--1995} 	{Técnico Electromecánico} 	{ENET Nº1 Brigadier General Pascual Echagüe}	{Concordia, Entre Ríos} {\textit{Promedio -- 8.5}} 							{}
%\cventry{1982--1989} 	{Escuela Primaria} 		{Escuela Velez Sarsfield}			{Concordia, Entre Ríos} {\textit{Promedio -- 8.5}} 							{}
%----------------------------------------------------------------------------------------
%	WORK EXPERIENCE SECTION
%----------------------------------------------------------------------------------------
\section{Experiencia}
	\subsection{Profesional}
	\cventry{2011--Presente}{Consultor y desarrollador de equipos electrónicos} 	{\href{www.seconsat.com} 		{Seconsat}} 		{}{}{Consultoría y desarrollo de accesorios electrónicos para el rubro AVL \href{http://disenioconingenio.com.ar/producto.php?products_id=392}{ver detalles}}
	\cventry{2005--Presente}{Director en empresa de ingeniería} 			{\href{www.disenioconingenio.com.ar} 	{disenioconingenio}} 	{}{}{Emprendimiento personal. Estudio de ingeniería que ofrece servicios de diseño electrónico a empresas, consultoría, soporte técnico, fabricación y puesta en marcha de placas OEM, y también ofrece productos electrónicos cerrados para integradores. \href{http://disenioconingenio.com.ar}{ver detalles}}
	\cventry{2003--2005} 	{Desarrollador de equipos electrónicos} 		{\href{www.digicard.com.ar} 		{Digicard}} 		{}{}{Empresa referente a nivel nacional en el rubro de control de accesos. Se trabajo en el desarrollo de un lector RFID de 125khz para la linea de controladores de accesos. Se participo en todas las etapas desde el requerimiento, diseño, layout, prototipo, puesta en marcha, firmware, documentación general y para producción. Actualmente es un producto comercializado activamente por la empresa. \href{http://disenioconingenio.com.ar/producto.php?products_id=393}{ver detalles}}
	\cventry{2002--2003} 	{Desarrollador de firmware para microcontroladores} 	{\href{www.pump-control.com.ar} 	{Pump-Control}} 	{}{}{Empresa dedicada principalmente al diseño, desarrollo y producción de controladores electrónicos para la distribución de hidrocarburos. Se trabajó en el área de desarrollo de firmware para microcontroladores de 8bits de la linea Atmel, implementando protocolos de comunicaciones, control de accesos, control de dispenser de combustible, etc. \href{http://disenioconingenio.com.ar/producto.php?products_id=391}{ver detalles}}

	\subsection{Investigación}
	\cventry{2009--2009}{Ayudante en el Centro de investigaciones de Láseres y Aplicaciones} 	{CITEDEF} 	{}{}{Se trabajó como ayudante del Dr. Jorge Codnia y la Lic. Laura Azcárate en el armado de un condensador de flujo láser para la generación de isótopos, y los primeros avances en un nuevo espectrómetro de masas de tiempo de vuelo}

	\subsection{Docencia}
	\cventry{2004--2004}{Curso intensivo de programación de FPGA de Altera usando Quartus II} 	{ITBA} 		{}{}{Se realizo un curso introductorio con actividades practicas usando una placa de evaluación de Altera. \href{http://disenioconingenio.com.ar/shop/docs/fpga.pdf} {ver material}}

	%\subsection{Otros Trabajos}
	%\cventry{1994--2001}{Instalación y mantenimiento de instalaciones eléctricas comerciales} 	{} 		{}{}{Se realizaron trabajos de electricidad, instalaciones eléctricas comerciales, reparaciones y mantenimiento general a clientes particulares}

\section{Tesis de grado}
\cventry{2004} 		{Diseño e implementación de una pantalla dinámica basada en 3200 lámparas de filamento con 16 escalas de grises y 20fps actualizable por ftp}{LampMatrix}{}{}{Bajo la tutela del Profesor Villamil, se diseñó y fabricó íntegramente una pantalla publicitaria basada en lamparas de filamento. \href{http://www.youtube.com/watch?v=Usx4YUNpknc}{ver video}, \href{http://disenioconingenio.com.ar/shop/docs/lampmatrix.pdf}{ver trabajo}.}

\section{Cursos y seminarios}
\cventry{2012} 		{Primeras jornadas de procesamiento de señales e imagenes}			{UTN, GIBIO EDE2008 Electronic Design Expo} 	{8hs}  	{\href{http://disenioconingenio.com.ar/producto.php?products_id=396}{ver certificado}} 	{}{}
\cventry{2008} 		{Conferencia sobre tecnologías inalámbricas de Digi RF}				{EDE2008 Electronic Design Expo} 		{6hs}  	{\href{http://disenioconingenio.com.ar/producto.php?products_id=394}{ver certificado}} 	{}{}
\cventry{2007} 		{Curso teórico práctico de serigrafía orientado a la fabricación de PCB's}	{32hs} 				 			{\href{http://disenioconingenio.com.ar/producto.php?products_id=389}{ver detalles}} 	{}{}
\cventry{2007} 		{Seminario de desempeño analogico usando microcontroladores Silabs}		{8hs} 				 			{\href{http://disenioconingenio.com.ar/producto.php?products_id=389}{ver detalles}} 	{}{}
\cventry{2006} 		{Lanzamiento microcontroladores Freescale RS08KA, acelerómetros y sensores} 	{8hs} 							{\href{http://disenioconingenio.com.ar/producto.php?products_id=384}{ver certificado}} 	{}{}
\cventry{2006} 		{Lanzamiento microcontroladores Freescale Coldfire 32 bits} 			{10hs}			 				{\href{http://disenioconingenio.com.ar/producto.php?products_id=384}{ver detalles}} 	{}{}
\cventry{2004} 		{Microprocesadores Rabbit y Dinamic C} 						{24hs}			 				{\href{http://disenioconingenio.com.ar/producto.php?products_id=386}{ver certificado}} 	{}{}
\cventry{2002} 		{Curso teórico práctico IA “Inteligencia Artificial”} 				{ITBA} 						{18hs} 	{\href{http://disenioconingenio.com.ar/producto.php?products_id=387}{ver certificado}} 	{}{}
\cventry{1995} 		{Curso de radio aficionado con obtención de licencia LU9JGM} 			{Radio Club Concordia (LU9JJ)} 			{48hs} 	{\href{http://disenioconingenio.com.ar/producto.php?products_id=388}{ver detalles}} 	{}{}
%----------------------------------------------------------------------------------------
%	AWARDS SECTION
%----------------------------------------------------------------------------------------
\section{Premios}
\cventry{2002}{Iniciación en I+D ITBA} 					{1\textsuperscript{er} Premio} {}{}{\em{Diseño y Simulación de una Unidad de Punto Flotante con estructura Pipeline Multi-Thread para procesadores de propósitos generales de alta performance} \em \href{http://disenioconingenio.com.ar/shop/docs/i+d_itba_2002.pdf}{ver mas}}
\cventry{2001}{Robots de lucha Battle Tek, ITBA ``ingenio en acción''}	{3\textsuperscript{er} Puesto} {}{}{\em{Robot Discotech} \newline \em{Se diseño y fabricó un robot de lucha basado en un disco giratorio de alta velocidad de rotación con 2 salientes filosas que impactan contra el adversario. \href{http://disenioconingenio.com.ar/producto.php?products_id=378}{ver mas}}}
%----------------------------------------------------------------------------------------
%	COMPUTER SKILLS SECTION
%----------------------------------------------------------------------------------------
\section{Investigaciones y publicaciones}
\cventry{2008}{Estudio de técnicas foto térmicas aplicadas a la medición de flujo gaseoso} 						{CITEDEF} 											{}{}{Se presentó bajo la tutela Dr. Francisco Manzano y como meta de aprobación de Optoelectrónica II. 	\href{http://disenioconingenio.com.ar/shop/docs/citedef_2008.pdf} 	{ver trabajo}}
\cventry{2003}{Design and Simulation of a pipeline-structured Floating Point Unit for high performance general purpose processors} 	{JAIIO 32\textsuperscript{as} Jornadas Argentinas de Informática e Investigación Operativa} 														{\href{http://disenioconingenio.com.ar/shop/docs/jaiio2003.pdf} 	{ver trabajo}} {}{}
\cventry{2003}{Selección del número de etapas óptimas en unidades de punto flotante con estructura pipeline} 				{CACIC, Congreso argentino de ciencias de la computación} 					 													{\href{http://disenioconingenio.com.ar/shop/docs/cacic2003.pdf} 	{ver trabajo}} {}{}
%----------------------------------------------------------------------------------------
%	LANGUAGES SECTION
%----------------------------------------------------------------------------------------
\section{Dominio de tecnologías}
	\subsection{Programas de computadora}
	\cvitem{Avanzado} 	{linux, crypsetup, vim, mutt, gnumeric, ssh, bash, screen, Slic3r, Pronterface, Mach3, LinuxCNC, Rhinoceros, RhinoCam, Orcad16(Design CIS,Layout,Pspice), Windows(XP,Seven,Server2003,Office2000), Flash MX, Borland C++ Builder, octave}
	\cvitem{Intermedio} 	{Allegro PCB Router, \LaTeX, OpenOffice, LibreOffice, Wireshark, Matlab, Mathcad, quemu}
	\cvitem{Básico} 	{Quartus II, Delphi, Eclipse, Kicad}

	\subsection{Lenguajes de programación}
	\cvitem{Avanzado} 	{C, Octave, assembler}
	\cvitem{Intermedio} 	{C++, Pascal, bash, makefiles}
	\cvitem{Básico} 	{Java, VHDL, \textsc{html}}

	\subsection{Microcontroladores, microprocesadores y FPGA}
	\cvitem{Avanzado} 	{Freescale(HC9S08 8b, HC11 8b, HCS12 16b), Atmel AtmegaXX 8b}
	\cvitem{Intermedio} 	{Rabbit 2200 8b, Freescale(Coldfire V2 32b, Kinetis 32b), Altera Flex 10K10}
	\cvitem{Básico} 	{Texas MSP430 16b, Scilabs 8b}
%----------------------------------------------------------------------------------------
\section{Idiomas}
\cvitemwithcomment{Español} 	{Oral/Lectura/Escritura Avanzado} 		{Lengua nativa}
\cvitemwithcomment{Inglés} 	{Oral/Lectura/Escritura Intermedio} 		{TOEIC 2005--785 \href{http://disenioconingenio.com.ar/producto.php?products_id=390}{ver certificado}}
\cvitemwithcomment{Hebreo} 	{Lectura Intermedio, Escritura/Oral Básico} 	{Escuela primaria hebrea completa}
%%----------------------------------------------------------------------------------------
%%	OTROS INTERESES SECTION
%%----------------------------------------------------------------------------------------
%\section{Deportes}
%\cventry{1983--2004}{Basquet}{J.N.Bialik-Concordia}{ITBA}{}{Entrenamiento continuo en Basquet desde categoría mosquito hasta formar parte del plantel universitario}
%\cventry{1994--Presente}{Ciclismo}{}{}{}{Competición en categoría cross country sub-23, competencia en categoría trialbike sub 30, ciclismo amateur al presente}
%
%%----------------------------------------------------------------------------------------
%%	INTERESTS SECTION
%%----------------------------------------------------------------------------------------
%\section{Otras actividades e intereses}
%\renewcommand{\listitemsymbol}{-~} % Changes the symbol used for lists
%\cvlistdoubleitem{Sistemas Embebidos}{Linux}
%\cvlistdoubleitem{Electrónica}{Microcontroladores}
%\cvlistdoubleitem{Correr}{ciclismo}
%\cvlistdoubleitem{Astronomía}{Filosofía}
%\cvlistdoubleitem{Física}{Historia de la ciencia}
%%----------------------------------------------------------------------------------------
%
\end{document}
