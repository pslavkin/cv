%----------------------------------------------------------------------------------------
%	PACKAGES AND OTHER DOCUMENT CONFIGURATIONS
%----------------------------------------------------------------------------------------
\documentclass[11pt,a4paper,sans]{moderncv} % Font sizes: 10, 11, or 12; paper sizes: a4paper, letterpaper, a5paper, legalpaper, executivepaper or landscape; font families: sans or roman

\moderncvstyle{classic} % CV theme - options include: 'casual' (default), 'classic', 'oldstyle' and 'banking'
\moderncvcolor{orange} % CV color - options include: 'blue' (default), 'orange', 'green', 'red', 'purple', 'grey' and 'black'

\usepackage[utf8]{inputenc} 		% UTF8 para meter acentos y enies
\usepackage[scale=0.75]{geometry} 	% Reduce document margins

%----------------------------------------------------------------------------------------
%	NAME AND CONTACT INFORMATION SECTION
%----------------------------------------------------------------------------------------
\firstname{Pablo}
\familyname{Slavkin} 
\title{Curriculum Vitae}
\address{Padilla 876 3\textsuperscript{o} 5\textsuperscript{o}, Capital Federal}{Buenos Aires, Argentina}
\mobile{(+54)(911) 6 243 3463}
\phone{(+54)(11) 4962 6373}
\email{pslakin@disenioconingenio.com.ar}
\homepage{www.disenioconingenio.com.ar/categoria.php?css=1\&categories_id=191}{Pablo Slavkin}
\photo[130pt][0.4pt]{pictures/cara_ascci} 

%----------------------------------------------------------------------------------------
\begin{document}
\makecvtitle % Print the CV title
%----------------------------------------------------------------------------------------
%	EDUCATION SECTION
%----------------------------------------------------------------------------------------
\section{Educación}

\cventry{2007--Presente}{Doctorado en Ingeniería}{UTN - Universidad Tecnología Nacional FRBA}{Buenos Aires}{\textit{Promedio -- 10 sobre 3 materias aprobadas + 3 finales adeudados}}{Mención Procesamiento digital de señales} 
\cventry{1996--2005}{Ingeniería Electrónica}{ITBA - Intitulo Tecnológico de Buenos Aires}{Buenos Aires}{\textit{Promedio -- 6.5}}{}
\cventry{1990--1995}{Técnico Electromecánico}{ENET Nº1 Brigadier General Pascual Echagüe}{Concordia, Entre Ríos}{\textit{Promedio -- 8.5}}{}
\cventry{1982--1989}{Escuela Primaria}{Escuela Velez Sarsfield}{Concordia, Entre Ríos}{\textit{Promedio -- 8.5}}{}

\section{Tesis de Grado}
\cventry{2004}{Diseño e implementación de una pantalla dinámica basada en 3200 lámparas de filamento con 16 escalas de grises y 20fps actualizable por ftp}{LampMatrix}{}{}{Bajo la tutela del Professor Villamil, se diseñó y fabrico integramente una pantalla publicitaria basada en lamparas de filamento \newline \href{http://www.youtube.com/watch?v=Usx4YUNpknc}{ver video} \newline \href{http://disenioconingenio.com.ar/shop/docs/lampmatrix.pdf}{ver trabajo}}
%----------------------------------------------------------------------------------------
%	WORK EXPERIENCE SECTION
%----------------------------------------------------------------------------------------
\section{Experiencia}
\subsection{Profesional}
\cventry{2005--Presente}{Director en empresa de ingenieria}{\href{www.disenioconingenio.com.ar}{disenioconingenio}}{}{}{Emprendimiento personal. Estudio de ingenieria que ofrece servicios de diseño electronico a empresas, consultoria, soporte tecnico, fabricacion y puesta en marcha de placas OEM, y tambien ofrece productos electronicos cerrados para integradores}
\cventry{2011--Presente}{Consultor y desarrollador de equipos electrónicos}{\href{www.seconsat.com}{Seconsat}}{}{}{Consultoría y desarrollo de accesorios electrónicos para el rubro AVL}
\cventry{2003--2005}{Desarrollador de equipos electronicos}{\href{www.digicard.com.ar}{Digicard}}{}{}{Empresa referente a nivel nacional en el rubro de control de accesos. Se trabajo principalmente en el desarrollo de un lector de tags RFID de 125khz para toda la linea de controladores de accesos. Se participo en todas las etapas desde el requerimiento, diseño, layout, prototipo y puesta en marcha, firmware para varias marcas de tags, documentacion general y para produccion, y actualmente es un producto comercializado activamente por la empresa.}
\cventry{2002--2003}{Desarrollador de firmware para microcontroladores}{\href{www.pump-control.com.ar}{Pump-Control}}{}{}{Empresa dedicada principalmente al disenio, desarrollo y produccion de controladores electronicos para la distribucion de hidrocarburos. Se Trabajo en el area de desarrollo de firmware para microcontroladores de 8bits de la linea Atmel, implementando protocolos de comunicaciones, control de accesos, control de dispenser de combustible, etc.}
%------------------------------------------------
\subsection{Investigacion}
\cventry{2009--2009}{Ayudante en el Centro de investigaciones de Laseres y Aplicaciones}{CITEDEF}{}{}{Se trabajo como ayudante del Dr.Jorge Codnia y Laura Azcarate en el armado de un condensador de flujo laser para la generacion de isotopos, y los primeros avances en un nuevo espectrometro de masas de tiempo de vuelo}
\subsection{Docencia}
\cventry{2004--2004}{Curso intensivo de programacion de FPGA de Altera usando Quartus II}{ITBA}{}{}{Se realizo un curso introductorio con actividades practicas usando una placa de evaluacion de Altera \newline \href{http://disenioconingenio.com.ar/shop/docs/fpga.pdf} {ver material}}
\subsection{Otros Trabajos}
\cventry{1994--2001}{Instalacion y mantenimiento de instalaciones electricas comerciales}{}{}{}{Se realizaron trabajos de electricidad, instalaciones elctricas comerciales, reparaciones y mantenimiento general a clientes particulares}
%----------------------------------------------------------------------------------------
%	AWARDS SECTION
%----------------------------------------------------------------------------------------
\section{Premios}
\cventry{2002}{Iniciación en I+D ITBA} 					{1\textsuperscript{er} Premio} {}{}{\em{Diseño y Simulación de una Unidad de Punto Flotante con estructura Pipeline Multi-Thread para procesadores de propósitos generales de alta performance} \em \newline \href{http://disenioconingenio.com.ar/shop/docs/i+d_itba_2002.pdf}{ver mas}}
\cventry{2001}{Robots de lucha Battle Tek, ITBA ``ingenio en accion''}	{3\textsuperscript{er} Puesto} {}{}{\em{Robot Discotech} \newline \em{Se disenio y fabrico un robot de lucha basado en un disco giratorio de alta velocidad de rotacion con 2 salientes filosas que impactan contra el adversario, entre otras armas. \newline \href{http://disenioconingenio.com.ar/producto.php?products_id=378}{ver mas}}}
%----------------------------------------------------------------------------------------
%	COMPUTER SKILLS SECTION
%----------------------------------------------------------------------------------------
\section{Programas de computadora}
\cvitem{Avanzado}{Linux, Vim, Rhinoceros, RhinoCam, Flash MX, Borland C++ Builder, Octave, Orcad Design CIS, Orcad Layout 16.0, Orcad Pspice, Microsoft Windows XP-Server2003, Microsoft Office, mutt, gnumeric, ssh, bash, screen}
\cvitem{Intermedio}{Allegro PCB Router, \LaTeX, OpenOffice, LibreOffice, quemu, Wireshark, Matlab}
\cvitem{Basico} {Quartus II, Delphi}

\section{Lenguajes de programacion}
\cvitem{Avanzado}{C, Octave, assembler, }
\cvitem{Intermedio}{C++, Pascal, bash, }
\cvitem{Basico} {Java, VHDL, \textsc{html} }

\section{Microcontroladores, microprocesadores y FPGA}
\cvitem{Avanzado}{Freescale HC9S08 8b, HC11 8b, HCS12 16b, Atmel AtmegaXX 8b }
\cvitem{Intermedio}{Rabbit 8b, Freescale Coldfire V2 32b, Kinetis 32b, Altera Flex 10K10 }
\cvitem{Basico} {Texas MSP430 16b, Scilabs 8b }

\section{Dominio de tecnologias de comunicacion}
\cvitem{Avanzado}{Stack TCP/IPv4:arp,icmp,ip,tcp,sntp,ntp,smtp,udp,802.1x,dhcp,ethernet, 802.15.1, }
\cvitem{Intermedio}{}
\cvitem{Basico} { }
%----------------------------------------------------------------------------------------
%	COMMUNICATION SKILLS SECTION
%----------------------------------------------------------------------------------------
\section{Investigaciones y publicaciones}

\cventry{2008}{Estudio de técnicas fototérmicas aplicadas a la medición de flujo gaseoso} 						{CITEDEF} 											{}{}{Se presento bajo la tutela Dr. Francisco Manzano y como meta de aprobacion de Optoelectronica II \newline 	  \href{http://disenioconingenio.com.ar/shop/docs/citedef_2008.pdf} 	{ver trabajo}}
\cventry{2003}{Design and Simulation of a pipeline-structured Floating Point Unit for high performance general purpose processors} 	{JAIIO 32\textsuperscript{as} Jornadas Argentinas de Informática e Investigación Operativa} 	{}{} 														 {\href{http://disenioconingenio.com.ar/shop/docs/jaiio2003.pdf} 	{ver trabajo}}
\cventry{2003}{Selección del número de etapas óptimas en unidades de punto flotante con estructura pipeline} 				{CACIC, Congreso argentino de ciencias de la computación} 					{}{} 														 {\href{http://disenioconingenio.com.ar/shop/docs/cacic2003.pdf} 	{ver trabajo}}
%----------------------------------------------------------------------------------------
%	LANGUAGES SECTION
%----------------------------------------------------------------------------------------
\section{Idiomas}
\cvitem{Español}{Lengua madre}
\cvitem{Inglés}{Oral/Lectura/escritura Intermedio}
\cvitem{Hebreo}{Lectura Intermedio, Escritura/Oral Básico}

%----------------------------------------------------------------------------------------
%	INTERESTS SECTION
%----------------------------------------------------------------------------------------
\section{Intereses afines}
\renewcommand{\listitemsymbol}{-~} % Changes the symbol used for lists
\cvlistdoubleitem{Sistemas Embebidos}{Linux}
\cvlistdoubleitem{Electrónica}{Microcontroladores}

%----------------------------------------------------------------------------------------
%	OTROS INTERESES SECTION
%----------------------------------------------------------------------------------------
\section{Otras actividades e intereses}
\renewcommand{\listitemsymbol}{-~} % Changes the symbol used for lists
\cvlistdoubleitem{Correr}{ciclismo}
\cvlistdoubleitem{Astronomía}{Filosofía}
\cvlistdoubleitem{Física}{Historia de la ciencia}
%----------------------------------------------------------------------------------------

\end{document}
